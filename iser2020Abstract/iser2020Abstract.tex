
%%%%%%%%%%%%%%%%%%%% author.tex %%%%%%%%%%%%%%%%%%%%%%%%%%%%%%%%%%%
%
% sample root file for your "contribution" to a contributed volume
%
% Use this file as a template for your own input.
%
%%%%%%%%%%%%%%%% Springer %%%%%%%%%%%%%%%%%%%%%%%%%%%%%%%%%%


% RECOMMENDED %%%%%%%%%%%%%%%%%%%%%%%%%%%%%%%%%%%%%%%%%%%%%%%%%%%

\documentclass[graybox]{svmult}

% choose options for [] as required from the list
% in the Reference Guide

\usepackage{mathptmx}       % selects Times Roman as basic font
\usepackage{helvet}         % selects Helvetica as sans-serif font
\usepackage{courier}        % selects Courier as typewriter font
\usepackage{type1cm}        % activate if the above 3 fonts are
                            % not available on your system
%
\usepackage{makeidx}         % allows index generation
\usepackage{wrapfig}
\usepackage{graphicx}        % standard LaTeX graphics tool
                             % when including figure files
\usepackage{multicol}        % used for the two-column index
\usepackage[bottom]{footmisc}% places footnotes at page bottom
\usepackage{marginnote}
\newcommand{\TODO}[1]{\marginnote{TODO: #1}}    % TODO Befehl
\usepackage{units,amsmath}
\usepackage{todonotes}
\usepackage{hyperref}
\hypersetup{
    colorlinks=true,
    linkcolor=blue,
    filecolor=magenta,      
    urlcolor=cyan,
}
\usepackage{url}
\usepackage{siunitx}
\usepackage{units}
\usepackage{subcaption}
% see the list of further useful packages
% in the Reference Guide

\makeindex             % used for the subject index
                       % please use the style svind.ist with
                       % your makeindex program

%\usepackage{cleveref}[2012/02/15]
%\crefformat{footnote}{#2\footnotemark[#1]#3}

\renewcommand{\O}{{\cal O}}
\renewcommand{\leadsto}{\rightsquigarrow}
\newcommand{\V}[1]{\text{\boldmath $#1$}}    % Format for "Vector"
\newcommand{\M}[1]{\V{#1}}                   % Format for "Matrix"

\newcommand{\R}{\mathbbm{R}}                 % set of real number
\newcommand{\N}{\mathbbm{N}}                 % set of natural numbers
\newcommand{\C}{\mathbbm{C}}                 % ...
\newcommand{\1}{\mathbbm{1}}                 % identity matrix


%%%%%%%%%%%%%%%%%%%%%%%%%%%%%%%%%%%%%%%%%%%%%%%%%%%%%%%%%%%%%%%%%%%%%%%%%%%%%%%%%%%%%%%%%

%%
% Motivation, Problem Statement, Related Work (one page)
% Technical Approach (one page)
% Results (one page)
% Experiments completed or scheduled (one page)
% Main experimental insights (one page)
% References (one page)
%%

\begin{document}

\title*{Extended Abstract:           L.U.N.A. - A Laser-Mapping Unidirectional Navigation Actuator} 
\author{Jasper Zevering, Anton Bredenbeck, Fabian Arzberger,
  Dorit Borrmann and Andreas N\"uchter}
% FIXME sort the authors
% Use \authorrunning{Short Title} for an abbreviated version of
% your contribution title if the original one is too long

%\and Name of Second Author \at Name, Address of Institute 
%\email{name@email.address}

%
% Use the package "url.sty" to avoid
% problems with special characters
% used in your e-mail or web address
%
\maketitle

%\abstract*{Each chapter should be preceded by an abstract (10--15 lines long) 
%that summarizes the content. The abstract will appear \textit{online} at 
%\url{www.SpringerLink.com} and be available with unrestricted access. This 
%allows unregistered users to read the abstract as a teaser for the complete 
%chapter. As a general rule the abstracts will not appear in the printed 
%version 
%of your book unless it is the style of your particular book or that of the 
%series to which your book belongs.
%Please use the 'starred' version of the new Springer \texttt{abstract} command 
%for typesetting the text of the online abstracts (cf. source file of this 
%chapter template \texttt{abstract}) and include them with the source files of 
%your manuscript. Use the plain \texttt{abstract} command if the abstract is 
%also to appear in the printed version of the book.}

\renewcommand{\floatpagefraction}{.9} 

\section{Motivation, Problem Statement, Related Work}

To foster new advances in applications for robots in areas that are too dangerous for humans, specifically for underground environments, the  Defense Advanced Research Projects Agency (DARPA) of the US Defense Department established the yearly ``SubT'' Challenge in 2017.
In this challenge, teams are tasked to ``Drive novel approaches and technologies to allow warfighters and first-responders to rapidly map, navigate, and search dynamic underground environments''~\cite{allen} proving the demand for further research in this domain.
One subtask of this challenge is building an accurate 3D model of the environment, i.e., mapping the surroundings. 
This paper shows a proof of concept of a such a novel approach and validates it with experiments.

One such approach using a 2D laser scanner to scan 3D indoor environments has been proposed in~\cite{classical_mechanics_scanner}.
The authors mount a 2D laser scanner on a cylindrical structure.
An operator then initiates a rolling motion by manually pushing the contraption.
This enables the scanner to sense the 3D environment successfully.
However, manually pushing the scanner is not practical, especially for long scans.
%
Previous work includes our RADLER (RADial LasER scanning device), which consists of a 2D laser scanner attached to the axle of a unicycle~\cite{ISER2018}.
An operator pushes the unicycle along a requested path.
The inherent rotation of the wheel creates a radial 3D laser scanning pattern.
However, this approach still requires an operator, therefore does not fulfill the autonomy requirements. 
%
A more autonomous approach was taken by Fang et al.~\cite{3D_per_2D_based}.
The authors mounted a rotating 2D laser-scanner on top of a turtle-bot thus removing the need of an operator.
In contrast to RADLER however, the turtle-bot does not provide an inherent rotation.
Therefore an additional actuator is required to create the radial 3D scanning-pattern. 

This paper builds upon the results of RADLER and has a specific application of mapping lunar craters autonomously in mind.
We propose a novel approach to low-cost 3D laser scanning using a 2D laser scanner inside a spherical robot based on impulse by conversation of angular momentum (IBCOAM): the L.U.N.A. - sphere (Laser-mapping Unidirectional Navigation Actuator).
The 2D laser scanner is fixed to the spherical structure, hence a similar situation as with the RADLER is given: the inherent rotation of the sphere creates a radial 3D scanning pattern.
Using the format of a spherical robot permits the system to be designed more compact. 
Additionally, the spherical shell doubles as a protective layer for the actuators, sensors and electronics. 
This is especially valuable for applications in rough terrain or scenarios in which non-minimal impact is expected, such as space applications.
During a launch, withstanding large G-forces is a necessary requirement, which can be better implemented using the spherical format. 
Furthermore, an operator is no longer required given a drive implemented in the robot.

The design of drive systems for self-driving spheres goes back as far as 1893 with a patent for a sphere driven by an inner moving counterweight, which got its torque from a spring~\cite{tate1893sphere}.
In the more recent past various approaches and their implementation have been presented which will be detailed in the final paper~\cite{soa1,soa2,soa3,soa4,soa5,soa6,soa7}.
In order to use the rotation axis of the sphere as an additional degree of freedom for the sensor, the sensor needs to be positioned close to the center of the sphere with as little occlusion as possible and the sensor cannot be used as part of a static counterweight inside the sphere.
This makes most of the existing approaches infeasible.
Inspired by the seminal Cubli \cite{cubliIROS12}, we started designing a spherical robot based on a impulse by conservation of momentum (IBCOAM) drive. 


\section{Technical Approach}
\label{sec:TechnicalApproach}

The L.U.N.A robot is a sphere robot. It carries a  laser-scanner which measures 541 points in one line. Therefore, when the sphere is rolling, L.U.N.A is capable to reconstruct the environment three-dimensional. To accomplish the rotation L..U.N.A has 2 flywheels.
As result of the impulse-momentum theorem, for a change of the momentum of the flywheels, an impulse, rooted in a force, is needed. 
The motors provide this force, and according to Newtons' third law, a force is applied back, resulting in the opposite change of momentum of the sphere. 
So the rotation of the sphere is not a direct consequence of the angular momentum of the flywheels, but rather of the impulse needed to spin them. 
Figure \ref{sec:TechnicalApproach:fig:setup} shows a CAD blueprint of the overall interior layout of the mechanical structure of the L.U.N.A sphere, ignoring the outside sphere, flywheels and wiring.
The LMS-100 laser scanner is mounted to the supporting structural components which are made of acrylic glass.
The two brushless motors were each placed on one side of the supporting structure. 
The motors are mounted with spacers, that leave room for the side IMU underneath one of the motors. 
Two other IMUs are placed in front of and beneath the laser to ensure coverage of all axes. 
The flywheels are a combination of brass for high weight and acrylic glass on the outer radius for transparency.

Figure \ref{sec:TechnicalApproach:fig:setup} shows the final hardware setup.
In order to reduce complexity with respect to the 3D-transformation calculations, the laser scanner was centered inside the spherical acrylic glass shell as precisely as possible.
This reduces the translational offset of the laser scanner with respect to the rotational motion of the sphere. 
                                   

Three separate IMUs keep track of the pose of the sphere. Each IMU is placed in such a way that the IMUs $z$-axis corresponds to one possible rotation axis of the sphere.
Therefore, each IMU is perpendicular to the other two.
Combining the axes measurements leads to a "virtual" IMU, which emulates being an IMU positioned at the center of the sphere. 
Hence, isolating the measurements of the resulting virtual IMU to only the rotation in the given axis.
The IMUs also ship with accelero\-meters that are used to determine the full pose of the sphere.
Each IMU calculates their pose separately, using a quaternion extended Kalman filter (QEKF).
However, combining those poses into one does not have any positive effect, but only makes the software more resource demanding and slow.
Thus only the pose of the bottom IMU's accelerometer is used to keep track of the pose.

A controller was implemented that measures the extend of the vibrations using standard deviations of the non-rotating axes of the IMU and adjusts the throttle of the motors accordingly.
Considering the translational velocity of the sphere in a controller is not possible.
The speed of the sphere is calculated by the rotational speed, which is why slippage of the sphere causes such a controller to not produce the desired motion. 

For the processing of the point cloud the 3D Toolkit (3DTK) was used.
Therefore only the time-stamped raw data of the IMUs and laser scanner is transferred and the estimation of the pose and the SLAM algorithm itself is performed externally.


\section{Results}
\label{sec:Results}

As proof of concept, the prototype of the L.U.N.A sphere for 3D mapping using the concept of impulse by conservation of angular momentum as a unidirectional drive to roll a 2D laser scanner in an IMU-equipped, pose-tracked spherical robot system, has been build and tested successfully. Potential for improvement and technical limitations have been identified.
The resulting sphere is shown in Figure~\ref{sec:TechnicalApproach:fig:setup}.
\begin{figure}
\centering                                                                                                                                                                                                        
\includegraphics[height=50mm]{../Media/BlueprintPNG.png}                                                                                                                                                      \\
\vspace{0.5cm}
\includegraphics[height=50mm]{../Media/sphereFullshellLeft.jpg}
\includegraphics[height=50mm]{../Media/sphereRightMotor.jpg}   
\\\vspace{0.5cm}
\begin{subfigure}[b]{0.32\textwidth}
	\centering
	\includegraphics[width=\textwidth]{../Media/FirstDecentMap}
	\caption{Test with limited motion and no exterior shell.}
	\label{sec:experimentalResults:3DLaserScanning:fig:firstpointcloud}
\end{subfigure}
\begin{subfigure}[b]{0.32\textwidth}
	\centering
	\includegraphics[width=\textwidth]{../Media/testScanWithTop}
	\caption{Test with limited motion and exterior shell.}
	\label{sec:experimentalResults:3DLaserScanning:fig:secondpointcloud}
\end{subfigure}
\begin{subfigure}[b]{0.32\textwidth}
	\centering
	\includegraphics[width=\textwidth]{../Media/RollingTestMap}
	\caption{Test with exterior shell and full motion.}
	\label{sec:experimentalResults:3DLaserScanning:fig:thirdpointcloud}
\end{subfigure}
\caption{Hardware setup and laser scanning results of the L.U.N.A sphere prototype. The Hardware including notches in the shell and friction granule (middle left). IMU (beneath supporting structure) and brushless motor including flywheel mass (above supporting structure)(middle right).}
\label{sec:TechnicalApproach:fig:setup}

\end{figure}


\section{Experiments completed}

All Experiments have been completed.

Multiple basic roll tests: as the driving mechanism is not the common approach, where the rotation of a non-homogenus mass-distribution leads to the rotation, several basic tests for the driving mechanism were conducted. 


Outdoor and indoor scanning tests: the environment scanning was tested indoor and outdoor. 

Stress test for the microcontroller: evaluating the amount of required tasks capable by the microcontroller itself  and therefore scaling the need of a server structure.

Single IMU vs triple IMU test: the difference between the use of a single IMU and the triple IMU approach presented in 
\ref{sec:TechnicalApproach} was evaluated.

\section{Main Experimental Insights}

The experiments lead to multiple hardware and software improvements.
The impulse by conservation of angular momentum drive accelerates the L.U.N.A. sphere reliably.
Figure \ref{sec:technicalApproach:fig:angvel} shows the angular acceleration of the whole sphere measured by the IMU system in one test run. 
Furthermore, it shows that the acceleration along the rotational axis of the flywheels rises while the accelerations along the other axes remain lower, albeit are noisy. 
However, it also shows the decrease in noise due to the combination of the IMU measurements.
The vibrations and tilt of the robot contribute to the velocities along the other axes.
The vibrations are results of inexact drilling of the flywheels such that there is an unbalance.
At the main test site the ground is a hard, clean and low friction concrete floor.
In such a scenario the vibrations add up and lead to slippage.
However, a rubber surface (a running track) absorb the vibrations, such that the acceleration process happens reliably.\newline
Furthermore the tests regarding the laser scanning showed the need of further improvement like re-positioning the laser scanner and the need for cutting windows into the sphere.

\begin{figure}
\centering
\begin{subfigure}{0.45\textwidth}
\includegraphics[width=\textwidth]{./plotsAndScripts/angVel-2020-01-29-16-14-54/imu1_ang_vel}
\caption{IMU 1: $z$-axis corresponds to sphere $x$-axis}
\label{sec:technicalApproach:fig:imu1_ang_vel}
\end{subfigure}\hfill
\begin{subfigure}{0.45\textwidth}
\includegraphics[width=\textwidth]{./plotsAndScripts/angVel-2020-01-29-16-14-54/imu2_ang_vel}
\caption{IMU 2: $z$-axis corresponds to sphere $z$-axis}
\label{sec:technicalApproach:fig:imu2_ang_vel}
\end{subfigure}\hfill\\

\begin{subfigure}{0.45\textwidth}
\includegraphics[width=\textwidth]{./plotsAndScripts/angVel-2020-01-29-16-14-54/imu3_ang_vel}
\caption{IMU 3: $z$-axis corresponds to sphere $y$-axis}
\label{sec:technicalApproach:fig:imu3_ang_vel}
\end{subfigure}\hfill
\begin{subfigure}{0.45\textwidth}
\includegraphics[width=\textwidth]{./plotsAndScripts/angVel-2020-01-29-16-14-54/merged_ang_vel}
\caption{Merged virtual IMU. Maps the $z$-axis of all other IMUs to the rotational axes.}
\label{sec:technicalApproach:fig:merged_ang_vel}
\end{subfigure}\hfill
\caption{Angular velocity measurements of singular IMUs and the combined IMU.}
\label{sec:technicalApproach:fig:angvel}
\end{figure}

\begin{acknowledgement}
The authors thank Dieter Ziegler and Sergio Montenegro for supporting our work and the Elite Network Bavaria for providing funding. 

\subsection*{Authors Note}
In an attempt to abide by the \href{https://www.go-fair.org/fair-principles}{Fair-Principles} of open science the authors provided all code developed and further information at their \href{https://github.com/fallow24/L.U.N.A}{GitHub} page.
\end{acknowledgement}

\bibliographystyle{plain}
\bibliography{andreas_publications}

\end{document}
