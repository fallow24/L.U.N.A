\section{Conclusions}
\label{sec:conclusions}

In this paper a new cost efficient approach to 3D lase scanning was proposed: The L.U.N.A. sphere. It uses a 2D laser scanner mounted inside a spherical robot and uses the inherent rotational movement to form a radial scanning pattern and hence create a 3D point cloud. The spherical robot is based on conversation of angular momentum and uses flywheels to drive the robot forward. 

The prototype developed for the tests in this paper was able to move in one direction reliably on soft surfaces (such as rubber), however had difficulties with slippage on hard and slippery surfaces. In regards of 3D scanning this paper delivered a proof of concept, even though the result remain unsatisfactory as of right now. The biggest issues to overcome are reflections of the laser scanner beams by the exterior shell and synchronization issues between the IMU system and the laser scanner. 

Before the application of such a robot is possible more work is required. This could include improving the field of view of the laser scanner and extending the robot to two dimensional movement control. This would then enable autonomous mapping of environments using the L.U.N.A. sphere. 
