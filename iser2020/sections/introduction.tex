\section{Introduction}
\label{sec:introduction}

In today's world, autonomous robots have found their way into everyday life in a variety of ways. 
This includes, but is not limited to, the vacuum cleaner that independently navigates one's living-room or mobile robots employed for exploration of areas that are too dangerous for humans.
To foster new advances in the latter, specifically for underground environments, the  Defense Advanced Research Projects Agency (DARPA) of the US Defense Department established the yearly ``SubT'' Challenge in 2017.
In this challenge, teams are tasked to ``Drive novel approaches and technologies to allow warfighters and first-responders to rapidly map, navigate, and search dynamic underground environments.''~\cite{allen} proving the demand for further research in this domain.
One subtask of this challenge is building an accurate 3D model of the environment, i.e., mapping the surroundings.
The teams that participate in the DARPA challenge take advantage of high-quality hardware, such as state-of-the-art 3D laser scanners and cameras, thus making their solutions rather expensive.
However, the demand for mapping-solutions in the low-cost sector is non-negligible.
\todo{This paper shows a proof of concept and validates it with experiments.}

One such approach using a 2D laser scanner to scan 3D indoor environments has been proposed in~\cite{classical_mechanics_scanner}.
The authors mount a 2D laser scanner on a cylindrical structure.
An operator then initiates a rolling motion by manually pushing the contraption.
This enables the scanner to sense the 3D environment successfully.
However, manually pushing the scanner is not practical, especially for long scans.

Previous work includes our RADLER (RADial LasER scanning device), which consists of a 2D laser scanner attached to the axle of a unicycle~\cite{ISER2018}.
An operator pushes the unicycle along a requested path.
The inherent rotation of the wheel creates a radial 3D laser scanning pattern.
However, this approach still requires an operator, therefore does not fulfill the autonomy requirements. 

A more autonomous approach was taken by Fang et al.~\cite{3D_per_2D_based}.
The authors mounted a rotating 2D laser-scanner on top of a turtle-bot thus removing the need of an operator.
In contrast to the RADLER however, the turtle-bot does not provide an inherent rotation.
Therefore an additional actuator is required to create the radial 3D scanning-pattern. 

This paper builds upon the results of the RADLER and has a specific application of mapping lunar craters autonomously in mind.
We propose a novel approach to low-cost 3D laser scanning using a 2D laser scanner inside a spherical robot based on conversation of angular momentum (COAM): the L.U.N.A. - sphere (Laser-mapping Unidirectional Navigation Actuator).
The 2D laser scanner is fixed to the spherical structure, hence a similar situation as with the RADLER is given: the inherent rotation of the sphere creates a radial 3D scanning pattern.
Using the format of a spherical robot permits the system to be designed more compact.
\todo{The sphere protects the actuators and sensoric... + regarding space blub}
Furthermore, an operator is no longer required given a drive implemented in the robot.
