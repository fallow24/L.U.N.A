\section{State of the Art}
\label{sec:stateOfTheArt}
As evolution of RADLER, a hand-driven radial scanner device, a self-driving spherical approach was chosen to ensure robustness and autonomy.  There have been several works in literature regarding approaches for spherical robots, of which an overview will be given in the following. 

\subsection{Spherical Robots}
\label{sec:stateOfTheArt:sphericalrobots}
An early idea of an self-driving sphere has been introduced by J.L. Tate in 1893 who claimed the patent  508 and 558 in  the U.S. for a sphere, driven by an inner moving counterweight, which got its torque from an spring. This idea of an actuator attached to an counterweight pointing to the bottom and therefore the torque being transferred to the sphere and moving it, is still a widespread approach for spherical robots.

In \cite{soa1} a basic motion control system for the BYQ-III is introduced. The BYQ-III has a mass of 25kg and a diameter of 600mm and its driving mechanism has been proposed in \cite{soa2} by S.Hanxu, X,Aiping, J.Qingxuan and W.Liangqing. It contains  a counterweight pendulum, four gyro actuators, providing movement for two axes and one IMU mounted on the gyros' case. There is no extra payload or sensor, nor would there be space for a centered measurement unit due to the centered counterweight. Therefore the counterweight leads to a steady movement, not relying on acceleration but on velocity of the actuators and therefore providing continuous speed.

A second spherical robot with its driving mechanism relaying on inner counterweight is presented in \cite{soa3}. This robot was designed for movement on a water surface and therefore has fins mounted to the shell orthogonal to the movement. Two actuators attached to the shell and the inner counterweight provide movement around one axis. In contrast to the BYQ-III a middle-centered  sensor would be possible, but this would have no movement relative to the surface, as it would be part of the relatively non moving inner counterweight. It also has steady, well controllable movement. 
The spheres presented in \cite{soa4} and \cite{soa5} provide a solution for a driving system which does not rely on a moving inner counterweight but uses internal reaction wheels to provide torque. This leads to theoretically having middle-centered space available, which would be rotating with respect to the surface. However, the prototype provided by V. Muralidharan et al. shows less controllability then counterweight driven spheres. Furthermore, this robot is driven by acceleration and not velocity which leads to limited movement capabilities.

A third approach for spherical robots relys on an internal unit which drives inside the sphere. A Design and control approach is provided in \cite{soa6} where a four-wheeled vehicle moves in the sphere to initiate rolling by moving the center of mass in the desired direction. This technical solution is capable of a nearly maximum size of possible payload in relation to the overall-size, but also does not provide a rotation of the sensor which would be needed for 3D laser scanning. Additionally, this provides good controllability, similar to the the counterweight driven approach. It is obviously not as stable regarding external perturbations or forces, as there is no fixed connection to the shell. This would make it not suitable for missions with extreme forces and unknown starting conditions like missions involving a rocket launch or a hard landing to get to the starting point. In a worst case scenario this could lead to a start with the unit being rotated by 180 degrees and therefore not being able to bring torque to the sphere. A similar situation could occur if the sphere was stuck to the environment and therefore the inner unit would try to perform a whole revolution (``looping'') inside the sphere, which would cause the car to fall on its back, i.e. a supine position.

Overcoming this shortage of the inner unit just relying on gravity to apply its force to the shell, \cite{soa7} introduced the approach of a rod, expanded by a spring to the maximum possible size and having a wheel on one side. The wheel generates the movement by also moving the center of mass. However, now the non-reversible supine position does not exist anymore. Even with the wheel at the top, it is still pressed to the shell by the spring and therefore is capable of maintaining its movement. Again this approach does not provide spin of a centered  placed sensor without further contraptions.
