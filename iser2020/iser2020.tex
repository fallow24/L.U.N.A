
%%%%%%%%%%%%%%%%%%%% author.tex %%%%%%%%%%%%%%%%%%%%%%%%%%%%%%%%%%%
%
% sample root file for your "contribution" to a contributed volume
%
% Use this file as a template for your own input.
%
%%%%%%%%%%%%%%%% Springer %%%%%%%%%%%%%%%%%%%%%%%%%%%%%%%%%%


% RECOMMENDED %%%%%%%%%%%%%%%%%%%%%%%%%%%%%%%%%%%%%%%%%%%%%%%%%%%
\documentclass[graybox,a4paper]{svmult}

% choose options for [] as required from the list
% in the Reference Guide

\usepackage{mathptmx}       % selects Times Roman as basic font
\usepackage{helvet}         % selects Helvetica as sans-serif font
\usepackage{courier}        % selects Courier as typewriter font
\usepackage{type1cm}        % activate if the above 3 fonts are
                            % not available on your system
%
\usepackage{makeidx}         % allows index generation
\usepackage{wrapfig}
\usepackage{graphicx}        % standard LaTeX graphics tool
                             % when including figure files
\usepackage{multicol}        % used for the two-column index
\usepackage[bottom]{footmisc}% places footnotes at page bottom
\usepackage{marginnote}
\newcommand{\TODO}[1]{\marginnote{TODO: #1}}    % TODO Befehl
\usepackage{units,amsmath}
\usepackage{todonotes}
\usepackage{hyperref}
\hypersetup{
    colorlinks=true,
    linkcolor=blue,
    filecolor=magenta,      
    urlcolor=cyan,
}
\usepackage{url}
\usepackage{siunitx}
\usepackage{units}
\usepackage{subcaption}
% see the list of further useful packages
% in the Reference Guide

\makeindex             % used for the subject index
                       % please use the style svind.ist with
                       % your makeindex program

%\usepackage{cleveref}[2012/02/15]
%\crefformat{footnote}{#2\footnotemark[#1]#3}

\renewcommand{\O}{{\cal O}}
\renewcommand{\leadsto}{\rightsquigarrow}
\newcommand{\V}[1]{\text{\boldmath $#1$}}    % Format for "Vector"
\newcommand{\M}[1]{\V{#1}}                   % Format for "Matrix"

\newcommand{\R}{\mathbbm{R}}                 % set of real number
\newcommand{\N}{\mathbbm{N}}                 % set of natural numbers
\newcommand{\C}{\mathbbm{C}}                 % ...
\newcommand{\1}{\mathbbm{1}}                 % identity matrix


%%%%%%%%%%%%%%%%%%%%%%%%%%%%%%%%%%%%%%%%%%%%%%%%%%%%%%%%%%%%%%%%%%%%%%%%%%%%%%%%%%%%%%%%%

%%
% Motivation, Problem Statement, Related Work (one page)
% Technical Approach (one page)
% Results (one page)
% Experiments completed or scheduled (one page)
% Main experimental insights (one page)
% References (one page)
%%

\begin{document}

\title*{L.U.N.A. - A Laser-Mapping Unidirectional Navigation Actuator} 
\author{Jasper Zevering, Anton Bredenbeck, Fabian Arzberger,
  Dorit Borrmann and Andreas N\"uchter}
% FIXME sort the authors
% Use \authorrunning{Short Title} for an abbreviated version of
% your contribution title if the original one is too long
\institute{All authors are with Informatics VII -- Robotics and
  Telematics, University of W\"urzburg, Am Hubland, 97074 W\"urzburg
  \email{borrmann@informatik.uni-wuerzburg.de %\\|  \email{jasper.zevering@stud-mail.uni-wuerzburg.de} \\ | \email{anton.bredenbeck@stud-mail.uni-wuerzburg.de}\\ |  \email{fabian.arzberger@stud-mail.uni-wuerzburg.de
  }
%\and Name of Second Author \at Name, Address of Institute 
%\email{name@email.address}
}
%
% Use the package "url.sty" to avoid
% problems with special characters
% used in your e-mail or web address
%
\maketitle

%\abstract*{Each chapter should be preceded by an abstract (10--15 lines long) 
%that summarizes the content. The abstract will appear \textit{online} at 
%\url{www.SpringerLink.com} and be available with unrestricted access. This 
%allows unregistered users to read the abstract as a teaser for the complete 
%chapter. As a general rule the abstracts will not appear in the printed 
%version 
%of your book unless it is the style of your particular book or that of the 
%series to which your book belongs.
%Please use the 'starred' version of the new Springer \texttt{abstract} command 
%for typesetting the text of the online abstracts (cf. source file of this 
%chapter template \texttt{abstract}) and include them with the source files of 
%your manuscript. Use the plain \texttt{abstract} command if the abstract is 
%also to appear in the printed version of the book.}

\abstract{
This paper proposes an autonomous approach to 3D mapping using the concept of impulse by  conservation of angular momentum (IBCOAM) as a unidirectional drive to roll a 2D laser scanner in an IMU equipped, pose-tracked spherical robot system.
An experimental prototype of the robot is introduced, giving details about the hardware.
The laser scanning results as well as the IBCOAM drive data that have been gathered using the prototype are analyzed, revealing  technical challenges.
}

\input{sections/introduction}

\input{sections/stateoftheart}

\section{Technical Approach}
\label{sec:TechnicalApproach}

The L.U.N.A robot is a sphere robot. It carries a  laser-scanner which measures 541 points in one line. Therefore, when the sphere is rolling, L.U.N.A is capable to reconstruct the environment three-dimensional. To accomplish the rotation L..U.N.A has 2 flywheels.
As result of the impulse-momentum theorem, for a change of the momentum of the flywheels, an impulse, rooted in a force, is needed. 
The motors provide this force, and according to Newtons' third law, a force is applied back, resulting in the opposite change of momentum of the sphere. 
So the rotation of the sphere is not a direct consequence of the angular momentum of the flywheels, but rather of the impulse needed to spin them. 
Figure \ref{sec:TechnicalApproach:fig:setup} shows a CAD blueprint of the overall interior layout of the mechanical structure of the L.U.N.A sphere, ignoring the outside sphere, flywheels and wiring.
The LMS-100 laser scanner is mounted to the supporting structural components which are made of acrylic glass.
The two brushless motors were each placed on one side of the supporting structure. 
The motors are mounted with spacers, that leave room for the side IMU underneath one of the motors. 
Two other IMUs are placed in front of and beneath the laser to ensure coverage of all axes. 
The flywheels are a combination of brass for high weight and acrylic glass on the outer radius for transparency.

Figure \ref{sec:TechnicalApproach:fig:setup} shows the final hardware setup.
In order to reduce complexity with respect to the 3D-transformation calculations, the laser scanner was centered inside the spherical acrylic glass shell as precisely as possible.
This reduces the translational offset of the laser scanner with respect to the rotational motion of the sphere. 
                                   

Three separate IMUs keep track of the pose of the sphere. Each IMU is placed in such a way that the IMUs $z$-axis corresponds to one possible rotation axis of the sphere.
Therefore, each IMU is perpendicular to the other two.
Combining the axes measurements leads to a "virtual" IMU, which emulates being an IMU positioned at the center of the sphere. 
Hence, isolating the measurements of the resulting virtual IMU to only the rotation in the given axis.
The IMUs also ship with accelero\-meters that are used to determine the full pose of the sphere.
Each IMU calculates their pose separately, using a quaternion extended Kalman filter (QEKF).
However, combining those poses into one does not have any positive effect, but only makes the software more resource demanding and slow.
Thus only the pose of the bottom IMU's accelerometer is used to keep track of the pose.

A controller was implemented that measures the extend of the vibrations using standard deviations of the non-rotating axes of the IMU and adjusts the throttle of the motors accordingly.
Considering the translational velocity of the sphere in a controller is not possible.
The speed of the sphere is calculated by the rotational speed, which is why slippage of the sphere causes such a controller to not produce the desired motion. 

For the processing of the point cloud the 3D Toolkit (3DTK) was used.
Therefore only the time-stamped raw data of the IMUs and laser scanner is transferred and the estimation of the pose and the SLAM algorithm itself is performed externally.



\section{Results}
\label{sec:Results}

As proof of concept, the prototype of the L.U.N.A sphere for 3D mapping using the concept of impulse by conservation of angular momentum as a unidirectional drive to roll a 2D laser scanner in an IMU-equipped, pose-tracked spherical robot system, has been build and tested successfully. Potential for improvement and technical limitations have been identified.
The resulting sphere is shown in Figure~\ref{sec:TechnicalApproach:fig:setup}.
\begin{figure}
\centering                                                                                                                                                                                                        
\includegraphics[height=50mm]{../Media/BlueprintPNG.png}                                                                                                                                                      \\
\vspace{0.5cm}
\includegraphics[height=50mm]{../Media/sphereFullshellLeft.jpg}
\includegraphics[height=50mm]{../Media/sphereRightMotor.jpg}   
\\\vspace{0.5cm}
\begin{subfigure}[b]{0.32\textwidth}
	\centering
	\includegraphics[width=\textwidth]{../Media/FirstDecentMap}
	\caption{Test with limited motion and no exterior shell.}
	\label{sec:experimentalResults:3DLaserScanning:fig:firstpointcloud}
\end{subfigure}
\begin{subfigure}[b]{0.32\textwidth}
	\centering
	\includegraphics[width=\textwidth]{../Media/testScanWithTop}
	\caption{Test with limited motion and exterior shell.}
	\label{sec:experimentalResults:3DLaserScanning:fig:secondpointcloud}
\end{subfigure}
\begin{subfigure}[b]{0.32\textwidth}
	\centering
	\includegraphics[width=\textwidth]{../Media/RollingTestMap}
	\caption{Test with exterior shell and full motion.}
	\label{sec:experimentalResults:3DLaserScanning:fig:thirdpointcloud}
\end{subfigure}
\caption{Hardware setup and laser scanning results of the L.U.N.A sphere prototype. The Hardware including notches in the shell and friction granule (middle left). IMU (beneath supporting structure) and brushless motor including flywheel mass (above supporting structure)(middle right).}
\label{sec:TechnicalApproach:fig:setup}

\end{figure}


\input{sections/conclusion}

\begin{acknowledgement}
The authors thank Dieter Ziegler and Sergio Montenegro for supporting our work. We acknowledge funding from the ESA Contract No. 4000130925/20/NL/GLC for the ``DEADALUS -- Descent And Exploration in Deep Autonomy of Lava Underground Structures'' Open Space Innovation Platform (OSIP) lunar caves-system study and the Elite Network Bavaria (ENB) for providing funds for the academic program ``Satellite Technology''.

\bigskip

\noindent \textbf{Authors Note}
In an attempt to abide by the \href{https://www.go-fair.org/fair-principles}{Fair-Principles} of open science the authors provided all code developed and further information at their \href{https://github.com/fallow24/L.U.N.A}{GitHub} page.
\end{acknowledgement}

\bibliographystyle{plain}
\bibliography{andreas_publications}

\end{document}
